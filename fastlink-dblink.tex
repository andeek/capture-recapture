\documentclass[12pt]{article}\usepackage[]{graphicx}\usepackage[]{color}
% maxwidth is the original width if it is less than linewidth
% otherwise use linewidth (to make sure the graphics do not exceed the margin)
\makeatletter
\def\maxwidth{ %
  \ifdim\Gin@nat@width>\linewidth
    \linewidth
  \else
    \Gin@nat@width
  \fi
}
\makeatother

\definecolor{fgcolor}{rgb}{0.345, 0.345, 0.345}
\newcommand{\hlnum}[1]{\textcolor[rgb]{0.686,0.059,0.569}{#1}}%
\newcommand{\hlstr}[1]{\textcolor[rgb]{0.192,0.494,0.8}{#1}}%
\newcommand{\hlcom}[1]{\textcolor[rgb]{0.678,0.584,0.686}{\textit{#1}}}%
\newcommand{\hlopt}[1]{\textcolor[rgb]{0,0,0}{#1}}%
\newcommand{\hlstd}[1]{\textcolor[rgb]{0.345,0.345,0.345}{#1}}%
\newcommand{\hlkwa}[1]{\textcolor[rgb]{0.161,0.373,0.58}{\textbf{#1}}}%
\newcommand{\hlkwb}[1]{\textcolor[rgb]{0.69,0.353,0.396}{#1}}%
\newcommand{\hlkwc}[1]{\textcolor[rgb]{0.333,0.667,0.333}{#1}}%
\newcommand{\hlkwd}[1]{\textcolor[rgb]{0.737,0.353,0.396}{\textbf{#1}}}%
\let\hlipl\hlkwb

\usepackage{framed}
\makeatletter
\newenvironment{kframe}{%
 \def\at@end@of@kframe{}%
 \ifinner\ifhmode%
  \def\at@end@of@kframe{\end{minipage}}%
  \begin{minipage}{\columnwidth}%
 \fi\fi%
 \def\FrameCommand##1{\hskip\@totalleftmargin \hskip-\fboxsep
 \colorbox{shadecolor}{##1}\hskip-\fboxsep
     % There is no \\@totalrightmargin, so:
     \hskip-\linewidth \hskip-\@totalleftmargin \hskip\columnwidth}%
 \MakeFramed {\advance\hsize-\width
   \@totalleftmargin\z@ \linewidth\hsize
   \@setminipage}}%
 {\par\unskip\endMakeFramed%
 \at@end@of@kframe}
\makeatother

\definecolor{shadecolor}{rgb}{.97, .97, .97}
\definecolor{messagecolor}{rgb}{0, 0, 0}
\definecolor{warningcolor}{rgb}{1, 0, 1}
\definecolor{errorcolor}{rgb}{1, 0, 0}
\newenvironment{knitrout}{}{} % an empty environment to be redefined in TeX

\usepackage{alltt}

%\pdfminorversion=4
% NOTE: To produce blinded version, replace "1" with "0" below.
\newcommand{\blind}{1}

\usepackage{xr-hyper}       % cross-referencing (must be loaded before hyperref)
\usepackage{hyperref}
\usepackage{amsmath}
% Preamble common to both main.tex and supplementary.tex

\usepackage{graphicx}
\usepackage{enumerate}
\usepackage{natbib}

% Custom packages
\usepackage{etoolbox} % table spacing
\usepackage{amsthm, amssymb}
\usepackage{booktabs}       % professional-quality tables
\usepackage{algorithm}      % algorithm environment
\usepackage{algpseudocode}
\usepackage{multirow}
\usepackage{sectsty}
\usepackage{tabularx}
\usepackage{graphicx}
\usepackage{tikz}           % vector graphics
\usepackage{bm}             % bold math symbols
\usepackage{xcolor}
\usepackage{microtype} 
\usepackage{import}
\usepackage{url}
\usepackage{soul, color}
\usepackage{titling}
\usetikzlibrary{arrows, snakes, backgrounds, patterns, matrix, shapes, fit, 
calc, shadows, plotmarks}

\graphicspath{{./figures/}}

% Support for multiple bibliographies
\usepackage{multibib}
\newcites{App}{References}

% Custom commands
\let\oldvec\vec
\renewcommand\vec{\bm}
\newcommand{\simfn}{\mathtt{sim}} % similarity function
\newcommand{\truncsimfn}{\underline{\simfn}} % truncated similarity function
\newcommand{\partfn}{\mathtt{PartFn}} % partition function
\newcommand{\distfn}{\mathtt{dist}} % distance function
\newcommand{\valset}{\mathcal{V}} % attribute value set
\newcommand{\entset}{\mathcal{R}} % set of records that make up an entity
\newcommand{\partset}{\mathcal{E}} % set of entities that make up a partition
\newcommand{\1}[1]{\mathbb{I}\!\left[#1\right]} % indicator function
\newcommand{\euler}{\mathrm{e}} % Euler's constant
\newcommand{\dblink}{\texttt{\upshape \lowercase{d-blink}}} % Name of scalable Bayesian ER model
\newcommand{\blink}{\texttt{\upshape \lowercase{blink}}} % Name of original Bayesian ER model
\newcommand{\secref}[1]{Section~\ref{#1}} % Section reference
\newcommand{\myparagraph}[1]{\smallskip\textbf{#1}}
\newcommand{\eat}[1]{}
\newcommand{\change}[1]{#1}
\DeclareMathOperator*{\argmax}{arg\,max}
%\newcommand{\change}[1]{\textcolor{blue}{#1}}
\def\spacingset#1{\renewcommand{\baselinestretch}%
  {#1}\small\normalsize} \spacingset{1}
  
\newtheorem*{remark}{Remark}
\newtheorem{proposition}{Proposition}
\newtheorem*{definition}{Definition}
\newtheorem*{theorem}{Theorem}
\newtheorem*{lemma}{Lemma}

% DON'T change margins - should be 1 inch all around.
\addtolength{\oddsidemargin}{-.5in}%
\addtolength{\evensidemargin}{-.5in}%
\addtolength{\textwidth}{1in}%
\addtolength{\textheight}{-.3in}%
\addtolength{\topmargin}{-.8in}%

\IfFileExists{upquote.sty}{\usepackage{upquote}}{}
\begin{document}

\def\spacingset#1{\renewcommand{\baselinestretch}%
{#1}\small\normalsize} \spacingset{1}

% ridiculous table spacing
\makeatletter
\AtBeginEnvironment{tabular}{%
  \def\baselinestretch{1}\@currsize}%
\makeatother


%%%%%%%%%%%%%%%%%%%%%%%%%%%%%%%%%%%%%%%%%%%%%%%%%%%%%%%%%%%%%%%%%%%%%%%%%%%%%%

\if1\blind
{
  \title{\bf Title}
  \author{Rebecca C. Steorts\thanks{
    This work was supported by the National Science Foundation through NSF-1652431 and NSF-1534412.}\hspace{.2cm}\\
    Department of Statistical Science and Computer Science, Duke University}
  \maketitle
} \fi

\if0\blind
{
  \bigskip
  \bigskip
  \bigskip
  \begin{center}
    {\LARGE\bf Predicting Voter Party Registration from Linked Data with Error Propagation}
\end{center}
  \medskip
} \fi

\bigskip

\begin{abstract}
Give abstract 
\end{abstract}

\noindent%
{\it Keywords:}  TBD
\vfill

\newpage
%\spacingset{1.5} % DON'T change the spacing!



\section{Introduction}
\label{sec:intro}
Motivation is removing duplications from large datasets such that post-processing can happen. Goal: scale to billions of databases, while providing some uncertainty propagation. 

I think we should create synthetic data that looks like the U.S. Census for a particular state for our experiments. For example, we can look at counties in Wyoming, which is the smallest state. The aggregate of course would be the entire state. 

\paragraph{Baseline}
\begin{enumerate}
\item Given the size of Wyoming, dblink can run all of it in order to give a baseline. 
\item We can also do a comparison to fastlink to give another baseline. 
\end{enumerate}


\paragraph{Algorithm 1}
\begin{enumerate}
\item An alternative that is needed at scale is the following: running dblink for each county. 
\item Then run fastlink using the output of dblink for all the counties. 
\item What is the precision, recall, population size, posterior standard error and how does this compare to the baselines? 
\item What is gained and lost? 
\end{enumerate}

\paragraph{Algorithm 2}
\begin{enumerate}
\item Run fastlink for the state, and remove links that are low probability. 
\item Then run dblink for the counties. What is the precision, recall, population size, posterior standard error and how does this compare to the baselines? 
\item What is gained and lost? 
\end{enumerate}

\paragraph{Questions}

\begin{enumerate}
\item Do we see a benefit over algorithm 1 versus algorithm 2? My intuition is that Algorithm 1 should be much better than Algorithm 2. 
\item We need to set up a synthetic data set to generate counties and states. We could try using FEBRL. 
\item It would be good to start with a very small synthetic data set with a made up region that is quite small that has sub-regions that are not very large to test out this proposed method before applying it to something more realistic. 
\item Other ideas for synthetic data generation?
\end{enumerate}

\clearpage
\newpage
\section{Related Work}

\section{Proposed Methodology}

\section{Evaluation Metrics}

\section{Data Sets}

\section{Results}

\section{Discussion}

\clearpage
\newpage

\bibliographystyle{apalike}
\bibliography{canonical}
\clearpage
\newpage


\end{document}
